\documentclass[14pt, usenames, dvipsnames, notes]{beamer}

\usetheme[numbering=fraction,progressbar=frametitle]{metropolis}

%\usepackage{tikz}
%\usetikzlibrary{arrows, arrows.meta, positioning, decorations.markings, shapes, snakes}
\usepackage{xcolor}
\usepackage[ngerman]{babel}
\usepackage[utf8]{inputenc}
\usepackage{multicol}
\usepackage{tabularx}
\usepackage{marvosym}
\usepackage{setspace}

\newcommand{\aussage}[1]{	\begin{frame}{Rollenspiel}
	\centering #1
	\end{frame}}

\newcommand\Wider[2][2.0cm]{%
\makebox[\linewidth][c]{%
  \begin{minipage}{\dimexpr\textwidth+#1\relax}
  \raggedright#2
  \end{minipage}%
  }%
}

\makeatletter
\def\blfootnote{\gdef\@thefnmark{}\@footnotetext}
\makeatother

\title{Open Educational Resources}
\subtitle{\hfill\& Bildungsgerechtigkeit}
\author{Valentin Roland \& Charlotte Bäcker}
\date{27.04.19}

\begin{document}
\maketitle

{
    \usebackgroundtemplate{
    \includegraphics[width=\paperwidth]{./img/Global_Open_Educational_Resources_Logo.pdf}
    }
    \begin{frame}{}
        \vspace{-.1cm}

        \blfootnote{\vspace{-.7cm}\textcolor{white}{OER Global Logo by Jonathas Mello (CC BY 3.0).}}
    \end{frame}
}

\begin{frame}{}
    \begin{minipage}{.45\textwidth}
        \centering
        \includegraphics[width=.8\textwidth]{img/serlo-logo.png}
    \end{minipage}
    \hfill
    \begin{minipage}{.45\textwidth}
        \centering
        \includegraphics[width=\textwidth]{img/mfnf.png}
    \end{minipage}
\end{frame}

%\setbeamercolor{background canvas}{bg=mDarkTeal}

\begin{frame}{1. Teil}
    \begin{center}
        \setstretch{1.7}
        {\Large Was ist OER? Was ist Bildungsgerechtigkeit?}

        \medskip

        {\Large Hängt das Zusammen?}
    \end{center}
\end{frame}

\begin{frame}{2. Teil}
    \begin{center}
        \setstretch{1.7}
        {\Large Wie entstehen OER? }

        \medskip

        {\Large Wie sehen Lehr- und Lernmaterialien zukünftig aus?}
    \end{center}
\end{frame}


\setbeamercolor{background canvas}{bg=white}

\end{document}
