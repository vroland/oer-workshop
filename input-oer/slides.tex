\documentclass[14pt, usenames, dvipsnames, notes]{beamer}

\usetheme[numbering=fraction,progressbar=frametitle]{metropolis}

%\usepackage{tikz}
%\usetikzlibrary{arrows, arrows.meta, positioning, decorations.markings, shapes, snakes}
\usepackage{xcolor}
\usepackage[ngerman]{babel}
\usepackage[utf8]{inputenc}
\usepackage{multicol}
\usepackage{tabularx}
\usepackage{marvosym}
\usepackage{setspace}

\newcommand{\aussage}[1]{	\begin{frame}{Rollenspiel}
	\centering #1
	\end{frame}}

\newcommand\Wider[2][2.0cm]{%
\makebox[\linewidth][c]{%
  \begin{minipage}{\dimexpr\textwidth+#1\relax}
  \raggedright#2
  \end{minipage}%
  }%
}

\makeatletter
\def\blfootnote{\gdef\@thefnmark{}\@footnotetext}
\makeatother

\title{Open Educational Resources}
\subtitle{}
\author{Valentin Roland \& Charlotte Bäcker}
\date{28.04.19}

\begin{document}
\maketitle

\section{Was sind \glqq OER\grqq?}

    {
        \usebackgroundtemplate{
        \includegraphics[width=\paperwidth]{./img/Global_Open_Educational_Resources_Logo.pdf}
        }
        \begin{frame}{}
            \vspace{-.1cm}

            \blfootnote{\vspace{-.7cm}\textcolor{white}{OER Global Logo by Jonathas Mello (CC BY 3.0).}}
        \end{frame}
    }

    \begin{frame}{UNESCO - Definition}
        \glqq \alert{teaching, learning and research materials} in any medium, [...], that reside in the \alert{public domain} or have been released under an \alert{open license} that permits \alert{no-cost access, use, adaptation and redistribution} [...] with no or limited restrictions.\grqq \footnote{UNESCO, \tiny{https://en.unesco.org/themes/building-knowledge-societies/oer\#collapseOne, 16.03.19}}%\footnote{UNESCO, \url{}}

        \note{
            \begin{itemize}
                \item Freie Lizenz
                \item Freie Verbreitung und Veränderung
                \item Freier Zugriff für alle
                \item Kommerzielle Benutzung erlaubt
            \end{itemize}
        }
    \end{frame}

    \begin{frame}{Alternativ: 5R Permissions}
        \hfill\begin{minipage}{.85\textwidth}
            \begin{itemize}
                \item[\alert{Retain}] {\small right to} make, own, control copies
                \item[\alert{Reuse}] {\small right to} use in wide range of contexts
                \item[\alert{Revise}] {\small right to} adapt, modify, alter the content
                \item[\alert{Remix}] {\small right to} combine with other materials
                \item[\alert{Redistribute}] {\small right to} share (original or altered)
            \end{itemize}
        \end{minipage}
        \blfootnote{by David Wiley (CC-BY 4.0), \tiny{http://opencontent.org/definition/}}

        \note{
            \begin{itemize}
                \item Retain: Aufrufen eines Wikipediartikels
                \item Reuse: Aufgabensammlung als Unterrichtsmaterial, Bilder in Vortrag
                \item Revise: Lehrtext übersetzen, an Zielgruppe anpassen
                \item Remix: Aufgabensammlung erstellen
                \item Redistribute: Selbsterklärend, tw. Zwang $\rightarrow$ Copyleft
            \end{itemize}
        }
    \end{frame}

    \begin{frame}{Video}
        \begin{center}
            Why Open Education Matters by David Blake (CC-BY 3.0)
        \end{center}
    \end{frame}

    \begin{frame}{Ziele}
        \begin{center}
            Bildungsgerechtigkeit\\
            \vspace{1cm}
             Größere Materialienvielfalt\hfill Kostensenkung\\
             \vspace{1cm}
            Qualitätssteigerung durch Zusammenarbeit
        \end{center}
        \note {
            \begin{itemize}
                \item Freier Zugang für Benachteiligte $\rightarrow$ Notwendig, nicht Hinreichend
                \item Größere Vielfalt $\Rightarrow$ Andere Herangehensweisen, Perspektiven
                \item Finanzierung aus öffentlicher Hand? $\Rightarrow$ Einmalkosten für materialien
            \end{itemize}
        }
    \end{frame}

\section{Freie Lizenzen}

    \begin{frame}{Lizenzen}
        \begin{center}
            \includegraphics[width=.7\textwidth]{img/cc-text.pdf}
        \end{center}
        \includegraphics[width=.4\textwidth]{img/GFDL_Logo.pdf}
        \hfill
        \includegraphics[width=.4\textwidth]{img/GPLv3_Logo.pdf}

        \note {
            GFDL - GNU Free Documentation License (GNU Project, Wikipedia)\\
            GPL - GNU General Public License (GNU Project, Free Software)\\
            $\hookrightarrow$ GFDL, GPL zwangsläufig Copyleft

            CC - Creative Commons, Modulare Lizenz, für OER größte Verbreitung
        }
    \end{frame}

    \begin{frame}{Creative Commons - Module}
        \begin{tabularx}{\textwidth}{m{1.5cm}|p{1.5cm}|X}
            \includegraphics[width=1cm]{img/Cc-by.pdf} & BY & Attribution\\

            \includegraphics[width=1cm]{img/Cc-sa.pdf} & SA & Share Alike\\

            \includegraphics[width=1cm]{img/Cc-nc.pdf} & NC & Non-Commercial\\

            \includegraphics[width=1cm]{img/Cc-nd.pdf} & ND & No Derivatives\\
        \end{tabularx}
    \end{frame}

\section{Lernplattformen}

	\begin{frame}{Lernplattformen - Frei oder nicht?}
        \begin{minipage}{.5\textwidth}
            \begin{itemize}
                \item frustfrei-lernen.de
                \item Khan Academy
                \item Leifi Physik
                \item Mathe für Nicht-Freaks
                \item memucho.de
            \end{itemize}
        \end{minipage}
        \begin{minipage}{.45\textwidth}
            \begin{itemize}
                \item schlaukopf.de
                \item schulminator.com
                \item Serlo.org
                \item The SimpleClub
                \item unterricht.de
                \item Wikipedia
            \end{itemize}
        \end{minipage}
	\end{frame}
	\begin{frame}{Lernplattformen - Frei oder nicht?}
        \begin{tabularx}{\textwidth}{c|c}
			\textbf{frei} & \textbf{nicht frei} \\
			\hline
			Mathe für Nicht-Freaks & frustfrei-lernen.de \\
			& Khan Academy \\
			memucho.de & Leifi Physik \\
			& schlaukopf.de \\
			Serlo.org & schulminator.com \\
			& The SimpleClub \\
			Wikipedia & unterricht.de \\
		\end{tabularx}
	\end{frame}

\section{Rollenspiel}

	\aussage{Du kannst dir Nachhilfe leisten, wenn du in der Schule etwas nicht verstanden hast.}

	\aussage{Wenn du dich für ein Thema besonders interessierst, findest du in der Schule und zuhause ausreichend Material, um dich weiter damit zu beschäftigen.}

	\aussage{Zuhause bekommst du Hilfe bei Hausaufgaben oder Problemen mit dem Schulstoff.}

	\aussage{Du hast ausreichend Zeit, dir selbst ergänzendes Material zu den Themen aus der Schule herauszusuchen und durchzuarbeiten, um passende Erklärungen und Übungsaufgaben zu finden.}

	\aussage{Deine Eltern kennen sich mit deinem Wunschberuf und dem Weg dahin aus oder können sich gegebenenfalls informieren und dich so auf deinem weiteren Lebensweg unterstützen.}

	\aussage{Du hast zuhause genügend freie Zeit zum Entspannen und gehst dadurch immer ausgeruht in die Schule und kannst dich vollkommen auf den Unterricht konzentrieren.}
	
	\aussage{Wenn du im Unterricht unter- oder überfordert bist, haben deine Lehrer immer zusätzliches Material dabei.}

\section{Diskussion}

%TODO: Add notes for all questions

\setbeamercolor{background canvas}{bg=mDarkTeal}

\begin{frame}{}
    \begin{center}
        \setstretch{1.7}
        \textcolor{white}{\LARGE Schafft OER mehr Bildungsgerechtigkeit?}
    \end{center}
    \note{
        879 Millionen Euro geben Eltern in Deutschland pro 
        Jahr für Nachhilfe aus, davon die Hälfte für Mathematik (Bertelsmann Stiftung 2016).

        453.000 deutschsprachige SekundarschülerInnen 
        leiden unter Armut und haben zugleich Probleme 
        in der Schule (bpb/Destatis Datenreport 2016; AWO-ISS 2012).

        Bildungschacengleichheit (Nicht-Diskriminierung) $\leftrightarrow$ Bildungsgerechtigkeit (Nachteilsausgleich)\\


        Persönliche Erfahrungen mit Bildungsgerechtigkeit?\\


        Und was braucht es noch?\\
        $\hookrightarrow$ Serlo Labschool (Sebstständig "Lernen Lernen")\\
        $\hookrightarrow$ Welche Zielgruppen Priorisieren? (z.B. Sero: Hauptschule zuerst)
    }
\end{frame}

\begin{frame}{}
    \begin{center}
        \setstretch{1.7}
        \textcolor{white}{\LARGE OER oder \glqq nur\grqq~kostenlos - Macht das einen Unterschied?}
    \end{center}
\end{frame}

\begin{frame}{}
    \begin{center}
        \setstretch{1.7}
        \textcolor{white}{\LARGE Sollte die Bundesregierung OER speziell fördern?}
    \end{center}
    \note{
        \begin{itemize}
            \item OER direkt nicht förderbar -> Forschung, Konferenzen, Infrastruktur, da kein Markteingriff
            \item Polen OER als Bedingung für Ausschreibung
            \item Vielfalt $\leftrightarrow$ Lehrplankompatibilität\\
            \item Qualitätssicherung?
            \item Kostengünstiger? Keine Verlagsgebüren, Steuerfinanzierung?
        \end{itemize}
    }
\end{frame}


\setbeamercolor{background canvas}{bg=white}

\section{Quellen}

\begin{frame}{}
    \nocite{*}
    \bibliographystyle{alphadin}

    \twocolumn
    \bibliography{literature}
\end{frame}

\end{document}
