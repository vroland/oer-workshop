\documentclass[14pt]{beamer}

\usetheme[numbering=fraction,progressbar=frametitle]{metropolis}

%\usepackage{tikz}
%\usetikzlibrary{arrows, arrows.meta, positioning, decorations.markings, shapes, snakes}
\usepackage{xcolor}
\usepackage[ngerman]{babel}
\usepackage[utf8]{inputenc}
\usepackage{multicol}
\usepackage{tabularx}
\usepackage{marvosym} 
\usepackage{comment}

\newcommand\Wider[2][2.0cm]{%
\makebox[\linewidth][c]{%
  \begin{minipage}{\dimexpr\textwidth+#1\relax}
  \raggedright#2
  \end{minipage}%
  }%
}

\makeatletter
\def\blfootnote{\gdef\@thefnmark{}\@footnotetext}
\makeatother

\title{Open Educational Resources}
\subtitle{}
\author{Valentin Roland \& Charlotte Bäcker}
\date{28.04.19}

\begin{document}
\maketitle

\section{Was sind \glqq OER\grqq?}

{
    \usebackgroundtemplate{
    \includegraphics[width=\paperwidth]{./img/Global_Open_Educational_Resources_Logo.pdf}
    }
    \begin{frame}{}
        \vspace{-.1cm}

        \blfootnote{\vspace{-.7cm}\textcolor{white}{OER Global Logo by Jonathas Mello (CC BY 3.0).}}
    \end{frame}
}

\begin{frame}{UNESCO - Definition}
    \glqq \alert{teaching, learning and research materials} in any medium, [...], that reside in the \alert{public domain} or have been released under an \alert{open license} that permits \alert{no-cost access, use, adaptation and redistribution} [...] with no or limited restrictions.\grqq \footnote{UNESCO, \tiny{https://en.unesco.org/themes/building-knowledge-societies/oer\#collapseOne, 16.03.19}}%\footnote{UNESCO, \url{}}

    \note{
        \begin{itemize}
            \item Freie Lizenz
            \item Freie Verbreitung und Veränderung
            \item Freier Zugriff für alle
            \item Kommerzielle Benutzung erlaubt
        \end{itemize}
    }
\end{frame}

\begin{frame}{Ziele}

\end{frame}

\section{Quellen}

\begin{frame}{}
    \nocite{*}
    \bibliographystyle{alphadin}

    \twocolumn
    \bibliography{literature}
\end{frame}

\end{document}
