\documentclass[14pt, usenames, dvipsnames, notes]{beamer}

\usetheme[numbering=fraction,progressbar=frametitle]{metropolis}

%\usepackage{tikz}
%\usetikzlibrary{arrows, arrows.meta, positioning, decorations.markings, shapes, snakes}
\usepackage{xcolor}
\usepackage[ngerman]{babel}
\usepackage[utf8]{inputenc}
\usepackage{multicol}
\usepackage{tabularx}
\usepackage{marvosym} 
\usepackage{setspace}

\newcommand\Wider[2][2.0cm]{%
\makebox[\linewidth][c]{%
  \begin{minipage}{\dimexpr\textwidth+#1\relax}
  \raggedright#2
  \end{minipage}%
  }%
}

\makeatletter
\def\blfootnote{\gdef\@thefnmark{}\@footnotetext}
\makeatother

\title{Open Educational Resources}
\subtitle{}
\author{Valentin Roland \& Charlotte Bäcker}
\date{28.04.19}

\begin{document}
\maketitle

\section{Was sind \glqq OER\grqq?}

    {
        \usebackgroundtemplate{
        \includegraphics[width=\paperwidth]{./img/Global_Open_Educational_Resources_Logo.pdf}
        }
        \begin{frame}{}
            \vspace{-.1cm}

            \blfootnote{\vspace{-.7cm}\textcolor{white}{OER Global Logo by Jonathas Mello (CC BY 3.0).}}
        \end{frame}
    }

    \begin{frame}{UNESCO - Definition}
        \glqq \alert{teaching, learning and research materials} in any medium, [...], that reside in the \alert{public domain} or have been released under an \alert{open license} that permits \alert{no-cost access, use, adaptation and redistribution} [...] with no or limited restrictions.\grqq \footnote{UNESCO, \tiny{https://en.unesco.org/themes/building-knowledge-societies/oer\#collapseOne, 16.03.19}}%\footnote{UNESCO, \url{}}

        \note{
            \begin{itemize}
                \item Freie Lizenz
                \item Freie Verbreitung und Veränderung
                \item Freier Zugriff für alle
                \item Kommerzielle Benutzung erlaubt
            \end{itemize}
        }
    \end{frame}

    \begin{frame}{Alternativ: 5R Permissions}
        \hfill\begin{minipage}{.9\textwidth}
            \begin{itemize}
                \item[\alert{Retain}] {\small right to} make, own, control copies
                \item[\alert{Reuse}] {\small right to} use in wide range of contexts
                \item[\alert{Revise}] {\small right to} adapt, modify, alter the content
                \item[\alert{Remix}] {\small right to} combine with other materials
                \item[\alert{Redistribute}] {\small right to} share (original or altered)
            \end{itemize}
        \end{minipage}
        \blfootnote{by David Wiley (CC-BY 4.0), \tiny{http://opencontent.org/definition/}}

        \note{
            \begin{itemize}
                \item Retain: Aufrufen eines Wikipediartikels
                \item Reuse: Aufgabensammlung als Unterrichtsmaterial, Bilder in Vortrag
                \item Revise: Lehrtext übersetzen, an Zielgruppe anpassen
                \item Remix: Aufgabensammlung erstellen
                \item Redistribute: Selbsterklärend, tw. Zwang $\rightarrow$ Copyleft
            \end{itemize}
        }
    \end{frame}

    \begin{frame}{Video}
        \begin{center}
            Why Open Education Matters by David Blake (CC-BY 3.0)
        \end{center}
    \end{frame}

    \begin{frame}{Ziele} 
        \begin{center}
            Bildungsgerechtigkeit\\
            \vspace{1cm}
             Größere Materialienvielfalt\hfill Kostensenkung\\
             \vspace{1cm}
            Qualitätssteigerung durch Zusammenarbeit
        \end{center}
        \note {
            \begin{itemize}
                \item Freier Zugang für Benachteiligte $\rightarrow$ Notwendig, nicht Hinreichend
                \item Größere Vielfalt $\Rightarrow$ Andere Herangehensweisen, Perspektiven
                \item Finanzierung aus öffentlicher Hand? $\Rightarrow$ Einmalkosten für materialien
            \end{itemize}
        }
    \end{frame}

\section{Freie Lizenzen}

    \begin{frame}{Lizenzen}
        \begin{center}
            \includegraphics[height=.4\textheight]{img/cc.pdf}
        \end{center}
        \includegraphics[width=.4\textwidth]{img/GFDL_Logo.pdf}
        \hfill
        \includegraphics[width=.4\textwidth]{img/GPLv3_Logo.pdf}

        \note {
            GFDL - GNU Free Documentation License (GNU Project, Wikipedia)\\
            GPL - GNU General Public License (GNU Project, Free Software)\\
            $\hookrightarrow$ GFDL, GPL zwangsläufig Copyleft

            CC - Creative Commons, Modulare Lizenz, für OER größte Verbreitung
        }
    \end{frame}

\section{Lernplattformen - Frei oder nicht?}

	\begin{frame}{Lernplattformen - Frei oder nicht?}

		\begin{itemize}
			\item frustfrei-lernen.de
			\item Khan Academy
			\item Leifi Physik
			\item Mathe für Nicht-Freaks
			\item memucho.de
			\item schlaukopf.de
			\item schulminator.com
			\item Serlo.org
			\item The SimpleClub
			\item unterricht.de
			\item Wikipedia
		\end{itemize}
	\end{frame}
	\begin{frame}{Lernplattformen - Frei oder nicht?}
		\begin{tabular}{c|c}
			\textbf{frei} & \textbf{nicht frei} \\
			\hline
			Mathe für Nicht-Freaks & frustfrei-lernen.de \\
			& Khan Academy \\
			memucho.de & Leifi Physik \\
			& schlaukopf.de \\
			Serlo.org & schulminator.com \\
			& The SimpleClub \\
			Wikipedia & unterricht.de \\
		\end{tabular}
	\end{frame}

\section{Diskussion}

%TODO: Add notes for all questions

\setbeamercolor{background canvas}{bg=mDarkTeal}

\begin{frame}{}
    \begin{center}
        \setstretch{1.7}
        \textcolor{white}{\LARGE OER oder \glqq nur\grqq~kostenlos - Macht das einen Unterschied?}
    \end{center}
\end{frame}

\begin{frame}{}
    \begin{center}
        \setstretch{1.7}
        \textcolor{white}{\LARGE Copyleft - Verbreitungsbremse oder sinnvoller Selbstschutz?}
    \end{center}
\end{frame}

\begin{frame}{}
    \begin{center}
        \setstretch{1.7}
        \textcolor{white}{\LARGE Wo kommt OER in der Schule der Zukunft zum Einsatz?}
    \end{center}
    \note{ 
        \begin{itemize}
            \item Vielfalt $\leftrightarrow$ Lehrplankompatibilität\\
            \item Qualitätssicherung?
            \item Selbständiges Lernen mit OER? $\rightarrow$ Medienkompetenz
            \item Kostengünstiger? Keine Verlagsgebüren, Steuerfinanzierung?
        \end{itemize}
    }
\end{frame}


\setbeamercolor{background canvas}{bg=white}

\section{Quellen}

\begin{frame}{}
    \nocite{*}
    \bibliographystyle{alphadin}

    \twocolumn
    \bibliography{literature}
\end{frame}

\end{document}
